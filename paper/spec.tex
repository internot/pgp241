\documentclass[runningheads,a4paper]{llncs}[2018/03/10]

\newif\ifanon
\anonfalse

\usepackage[T1]{fontenc}
\usepackage[utf8]{inputenc}
\usepackage{amsmath}
\usepackage{color}
\usepackage{paralist}
\usepackage{xspace}
\usepackage{syntax}
\usepackage{balance}
\usepackage{graphicx}
\usepackage{listings}
\usepackage{enumitem}
\usepackage{csquotes}
\usepackage{xcolor}

\makeatletter
\newcommand{\shorteq}{%
  \settowidth{\@tempdima}{-}% Width of hyphen
  \resizebox{\@tempdima}{\height}{=}%
}
\makeatother

\newcommand{\todo}[1]{\textbf{\color{blue}TODO: #1}}

\definecolor{linkcolor}{rgb}{0,0,1}
\definecolor{citecolor}{rgb}{0,0,1}
\definecolor{urlcolor}{rgb}{0,0,1}
\usepackage[colorlinks=true, backref=page, linkcolor=linkcolor, urlcolor=urlcolor, citecolor=citecolor]{hyperref} 

\begin{document}
\ifanon
\else
	\author{Jonas Magazinius Nadim Kobeissi}
	\institute{
		Assured\\
		\email{jonas.magazinius@assured.se}
	}
	% I've not left you out, I just couldn't figure out how to add multiple authors without fucking up the layout and didn't want to spend time on it.
\fi

\title{OpenPGP 2-4-1: Merging two messages in one ciphertext}
\maketitle

\begin{abstract}
	{ASD}
\end{abstract}

\begin{keywords}
	OpenPGP
\end{keywords}

\section{Introduction}\label{sec:intro}
This is intro!

\section{Background}\label{sec:background}


\section{The OpenPGP message format}\label{sec:openpgp}

A PGP message consists of a set of packets. Each packet has a header consisting 
of a one byte tag and a variable
number of bytes describing the total size of the packet. Five kinds of
packets are of particular interest for this article, Literal,
Compressed, Symmetrically Encrypted (SE), Symmetrically Encrypted
Integrity Protected (SEIP), and Modification Detection Code (MDC)
packets. For a complete description of the PGP message format the
reader is referred to the PGP standard documentation, RFC4880.

The OpenPGP message format is specified and refined in RFCs 1991,
2440, and finally standardised in RFC
4880~\cite{callas1998rfc,callas2007openpgp}.

Of particular interest are symmetrically encrypted packets, both with
and without integrity protection.

\subsection{CFB mode}

In standard CFB mode, the first block of plaintext is XOR:ed with the encrypted 
IV. All consecutive plaintext blocks are XOR:ed with the ciphertext of the 
previous block.

$
\\
C_1 = E_K(IV) \oplus P_1 \\
C_i = E_K(C_{i-1}) \oplus P_i \forall i = 2 ... n
$


\subsubsection{Properties of CFB mode}

To start out, let us note some important properties concerning modification of a 
ciphertext encrypted in CFB mode. It is important to note that neither of these 
properties depend on the encryption key, which implies that knowing the key is 
not required to abuse them. The properties have historically been abused, and 
will again be abused in Section X, to attack the PGP message format.



\paragraph{Property 1} The ciphertext will have the exact length of
the plaintext and vice versa. A modification of the length of the
ciphertext will cause the same effect on the plaintext. Any number of
bytes cut off the end of the ciphertext will cut the same amount of
bytes from the plaintext.
 
\paragraph{Property 2} The decryption of a block depends only on the
preceding block, regardless of where in the ciphertext it
appears. Blocks can also be injected amidst two blocks, affecting only
the first block of the injected sequence and the following block after the sequence. As a result, any
sequence of ciphertext blocks can be cut, duplicated or moved between
blocks in the ciphertext, and still produce the same plaintext. The
decrypted output of the first block of the two and the block following
the second will however be garbled. This means an attacker can inject
these two blocks amidst blocks in the ciphertext and the decrypted
output will be one block of random data and one with arbitrary
attacker chosen text.


\paragraph{Property 3} When decrypting, the plaintext stands in direct
relation to the ciphertext. A flipped bit in the ciphertext causes the
same bit to be flipped in the plaintext. By extension, if the
plaintext of a block, or part of a block is known, that part of the
plaintext can be arbitrarily and reliably modified to produce any
text. Of course, modifying the ciphertext will cause the plaintext of
the following block to be scrambled. If the modified block is the last
block of the sequence, there will be no block to scramble.


\subsection{OpenPGP CFB mode}

OpenPGP implements CFB mode slightly differently compared to standard 
CFB mode \todo{Reference to RFC4880 Section 13.9}. There is also a slight difference in how an "Symmetrically 
Encrypted"-packet is encrypted compared to a "Symmetrically Encrypted Integrity 
Protected"-packet. For the purpose of this paper we will use the "Symmetrically 
Encrypted"-packet, and will therefore not go into the details of these 
differences.


Instead of encrypting an IV, the first block of ciphertext is formed of an block of zeroes is encrypted and XORed with a random 
number the size of a block, acting as an IV. Then the first block of ciphertext $C_{IV}$ is encrypted and the two first bytes of the output is XOR:ed with the two last bytes of the IV repeated, to form an IV "quick check". The encryption is then "resynced"

$
\\
C_{IV} = E_K(0) \oplus R \\
|C_{Check}|_{[1,2]} = |E_K(C_{IV})|_{[1,2]} \oplus |R|_{[15,16]} \\
C_1 = E_K(|C_{IV}|_{[3-16]}~~||~~|C_{Check}|_{[1,2]}) \oplus P_1 \\
C_i = E_K(C_{i-1}) \oplus P_i,~~\forall i = 2 ... n
$


\section{Two for one}\label{sec:twoforone}

The "Two for One" attack merges two messages, encrypted with different session 
keys, 
into a single ciphertext. The plaintext, when decrypted with the first key, 
corresponds to a Literal-packet followed by a Marker-packet. When decrypted with 
the second 
key, the first packet is the Marker-packet and the second the Literal packet. 
Since, upon decryption, the Marker-packet will be ignored, only the contents of 
the Literal-packet will be returned. This allows an attacker to send an 
encrypted message to different recipients, each receiving different keys, which 
will decrypt to different texts.



This works because the two decrypted plaintexts both contains two representations 
of packets. The plaintexts agree on the size and location of the packets, but 
disagree on their type. Apart 
from the packet header, the Marker-packets in both cases contain random garbage 
data. 

\subsection{Conditions}\label{sec:conditions}

Two conditions need to be met in order for the 

\paragraph{Quick-check bytes}
The last two bytes of the IV is repeated to form two "quick check" bytes. When 
decrypting, the two bytes must match or else the decryption fails. In the 
two-for-one scenario this means that we need to find two keys that generate output such that:

$E_{K_1}(0)_{[15,16]} \oplus E_{K_1}(C_{IV})_{[1,2]} \equiv E_{K_2}(0)_{[15,16]} 
\oplus E_{K_2}(C_{IV})_{[1,2]}$


\paragraph{Packet headers}

The output of the Packet headers must be valid and match two conditions. The 
types must be 

\subsection{In practice}\label{sec:in-practice}

Because the first ciphertext block is the XOR between the outputs $E_{K_i}(0)$ 
and random IVs, the ciphertext for the first block can be chosen at will. 


Steps:
\begin{enumerate}

	\item 1. select a ciphertext for the first block $C_{IV}$.
	\item 2. Randomly generate a key $K_i$ and compute $E_{K_i}$
3. Store $K_i$ 
4. Check if store contains a $K_j$ that fulfils the conditions.

\end{enumerate}
Once a matching $C_{IV}$, $K_i$ and $K_j$


\section{Conclusion}\label{sec:conclusion}
Well, wasn't this awesome!

\section*{Acknowledgments}
\ifanon
	Removed for blind review.
\else
	
\fi

\clearpage

\bibliographystyle{unsrt}
\bibliography{spec}


\end{document}
